\documentclass[aps,onecolumn]{revtex4}
%\setlength{\oddsidemargin}{-.38in}
%\setlength{\textheight}{9.6in}
\setlength{\textheight}{10.0in}
%\setlength{\textwidth}{6.7in}
%\setlength{\topmargin}{-0.3in}
\usepackage{bm}
\usepackage{array}
\usepackage{color}
\usepackage{rotating}
\usepackage{lscape}
\usepackage{graphicx}
\usepackage{amsmath}
\usepackage{amsfonts}
\usepackage{amssymb}
\usepackage{mathtools}
\usepackage{cases}
\usepackage{tikz,pgfplots}
\pgfplotsset{compat=1.10,width=12cm,
legend style={at={(0.03,0.5)},anchor=west}}
%This defines the interval between lines. To generate
%doublespaced text set it to 2.0
%\renewcommand{\baselinestretch}{1.92}

\renewcommand{\baselinestretch}{1.92}
\newcommand{\tr}{\mathop{\rm tr}\nolimits}
\renewcommand{\d}{\mathop{\rm d \!\,}\nolimits}
\renewcommand{\vec}{\mathop{\rm vec}\nolimits}
\newcommand{\vech}{\mathop{\rm vech}\nolimits}
%Key 1
\begin{document}
\title{Very accurate rovibrational transitions of
HeH$^+$ in the ground electronic state}
\author{
Keith Jones$^{a}$, Monika Stanke$^{b}$, Ludwik Adamowicz$^{a,c}$}
\affiliation{
$^{a}$Department of Chemistry and Biochemistry,
University of Arizona, Tucson, Arizona 85721, U.S.A.
\nonumber\\
$^{b}$ Institute of Physics, Faculty of Physics, Astronomy, and Informatics,
Nicolaus Copernicus University,
ul. Grudzi\c{a}dzka 5, Toru\'{n},
PL 87-100, Poland
\nonumber\\
$^{c}$Department of Physics,
University of Arizona, Tucson, Arizona 85721, U.S.A.}

%Key 0 - abstract
\begin{abstract}
HeH$^+$ rovibrational transitions are calculated using a very accurate Born-Oppenheimer
potential energy curve (PEC), which is generated in this work using variational calculations
performed in the basis set of explictly correlated Gaussian functions with shifted
centers. For each PEC point the exponential parameters of the Gaussins are optimized
using a method employing the analytic energy gradient determined with respect
to these parameters. The PEC energies are augmented 
with adiabatic corrections
for the nuclear motion and with the leading relativistic corrections for 
the motion of the electrons. The non-adiabatic corrections are also
approximately accounted for. A comparison is made
with the leading experimental data.
\end{abstract}

\maketitle

%Key 1 - Intro
\section{Introduction}
\label{Introduction}
HeH$^+$ is the lightest bound heteronuclear diatomic molecule. It is composed of the
two most abundant elements in the universe. The presence of HeH$^+$ in interstellar space and
planetary nebulae is expected \cite{Astro3, Astro4, Astro5}, and steady state abundances
of the molecule under different conditions have been calculated \cite{Astro1_SS_HeH+} although
it has yet to be observed \cite{Astro2, PES5}.

The existence of HeH$^+$ was confirmed in 1925 by mass spectrometry of discharges
containing He and H \cite{exp3}. Optical measurements would follow decades later.
HeH$^+$ has a permanent dipole moment, and very
accurate transition energies can be measured because of this. In 1979, the first rovibrational
transitions for HeH$^+$  ever measured (five transitions in total) were in the fundamental
band and in the first hot band of the electronic ground state \cite{exp2}. The
measurement had uncertainties
of 0.002 cm$^{-1}$. In the following decades, many more pure rotational \cite{exp1, exp7, exp9}
and rovibrational \cite{exp4, exp5, exp6, exp8, exp9, exp10} transitions of
HeH$^+$ were measured. The measurements were also done for
multiple HeH$^+$ isotopologues \cite{exp1, exp4, exp10}. Many of the later measurements were performed
using tunable diode lasers \cite{exp7, exp8, exp10}. There were also transitions between bound
and quasibound levels measured \cite{exp5}.

%1.) 1997 - pure rotational transitions of $^4$HeH$^+$, $^4$HeD$^+$, $^3$HeH$^+$, and
%    $^3$HeD$^+$ in the vibrational ground state \cite{exp1}.
%2.) 1989 - 54 total transitions spanning all four of the common isotopologues in either the
%    fundamental or first hot band. Uncertainties between 0.001 and 0.005 cm$^{-1}$ \cite{exp4}.
%3.) 1983 - Many different rovibrational transitions measured, including bound to quasibound
%    transitions \cite{exp5}.
%4.) 1982 - 9 rovibrational transitions measured in the fundamental band \cite{exp6}.
%5.) 1987 - First time using a tunable diode laser for rotational spectrocopy of molecular ions.
%    J=7<-6 transition in the vibrational ground state of HeH$^+$ \cite{exp7}.
%6.) 1987 - Tunable diode lasers used to measure rovibrational transitions of HeH$^+$ in the
%    fundamental and first hot band \cite{exp8}.
%7.) 1997 - Pure rotational transitions with $\nu$ = 0, 1, 2 and high J (10-21) up to the
%    dissociation limit were measured. 10 new rovibrational levels also measured up to the
%    dissociation limit \cite{exp9}.
%8.) 1992 - Rovibrational transitions in the $\nu$ = 0<-1, $\nu$ = 1<-2, $\nu$ = 2<-3, and
%    $\nu$ = 3<-4 bands were measured for HeH$^+$ and HeD$^+$ with a tunable diode laser
%    \cite{exp10}.

It is important for computational models to keep up with the accuracy of experimental
measurements of the rovibrational HeH$^+$ transitions 
in order for the unobserved transitions in this system calculated with
those models to be assumed correct. The most common way to model spectra of diatomics is to construct
a potential energy curve (PEC) being a function of the internuclear distance, and then solving the
Schr$\ddot{o}$dinger equation for the nuclear motion to acquire the rovibrational levels.

The first calculations of the potential energy curve for HeH$^+$ 
in the ground electronic state 
that was done accurately used a 64-term
generalized James-Coolidge expansion for the wave function \cite{PES7}. 
The calculation allowed the
approximation of the equilibrium bond length and the dissociation energy. 
Later, a ground-state potential
energy curve of HeH$^+$ between 0.9 and 9~a.u. was constructed using an 83-term variational
wave function expansion in elliptic coordinates with linear and non-linear variational
parameters fully optimized \cite{PES3, PES4}. The PEC was later improved by using a larger basis set and
by including adiabatic corrections \cite{PES8}. In contrast to the standard approach
based o2n the quantum mechanical calculations, a potential energy
curve for HeH$^+$ was also created by fitting using
experimental transitions and 19 tunable parameters \cite{PES2_exptrans_fit}. Recently,
a very accurate Born-Oppenheimer (BO) potential energy surface was constructed for HeH$^+$
in a basis of "asymptotically correct generalized Heitler-London functions"\cite{PES6},
followed by adiabatic, non-adiabatic, relativistic, and radiative corrections, which allowed
ervice0 very accurate calculation of the rovibrational transitions \cite{PES5}.

Explicitly correlated Gaussian functions (ECGs) were used to describe HeH$^+$ with accurate
single point calculations carried out for different electronic states \cite{ECG_HeH+} at
the ground-state equilibrium distance. In 2012 the ECGs were also used
to calculate a HeH$^+$ ground-state potential energy curve. The cureve incorporated
adiabatic and non-adiabatic corrections \cite{HeH+_Wei2012}.
ECGs have also been used to describe HeH$^+$ without assuming the Born-Oppenheimer (BO) approximation.
All rotationless vibrational states in the ground electronic were calculated \cite{Theory3}
with accounting for 
the leading relativistic and radiative corrections \cite{Theory4, Theory2}. 
This was also done using 
complex Gaussian basis functions \cite{Theory1}. The non-BO wave functions are 
functions of the coordinates 
of both electrons and nuclei and can be used to glean insight into the 
molecular structure that is
not possible from the BO description.

In this work, we introduce the leading
relativistic and radiative corrections to the PEC of HeH$^+$ created
in a similar way to the PEC generated in \cite{HeH+_Wei2012} 
in order to calculate the 
rovibrational transitions in the ground electronic state with
increased accuracy. The results are compared with 
the best available experimental data.

%MAYBE WHEN DISCUSSING WORK THAT HAS BEEN DONE, MENTION THE ORDER TO WHICH THE REL. RAD. CORRECTIONS
%HAVE BEEN DONE FOR EACH PARTICULAR STUDY.
%MAYBE MENTION ACCURACY OF DIFFERENT CALCULATSION IN CM-1

\section{Methodology}
\label{MTD}
\subsection{Optimization and basis set}
The electronic wave function for each point on the potential energy curve is a
linear expansion in terms of explicitly correlated Gaussian basis functions with shifted centers.
The Gaussians have the following
functional form:
\begin{eqnarray}
\phi_k(\mathbf{r}) = exp\Big[-(\mathbf{r - s_k})'(A_k\otimes I_3)(\mathbf{r - s_k})\Big], 
\label{ECG}
\end{eqnarray}
where $\mathbf{r}$ is a $3n$-dimensional vector of electronic coordinates of the form [x$_1$,
$y_1$,$z_1$,$x_2$,...,$z_n$] and $n$ is the number of electrons in the system, $\mathbf{s_k}$
is a $3n$ dimensional vector of the coordinates of the Gaussian shifts, $A_k$ is an $n \times n$ 
matrix of non-linear exponential. Matrix elements of both $\mathbf{s_k}$ and $A_k$ are 
optimization parameters.
In (\ref{ECG}) $\otimes$ represents the Kronecker product operation, and
$I_3$ is the $3 \times 3$ identity matrix.
Prime, '''", denotes the vector/matrix transposition.

During the variational optimization, the linear expansion coefficients 
of the wave function 
are evaluated by solving the secular equation:
\begin{eqnarray}
(\mathbf{H} - E\mathbf{S})\mathbf{c}.
\end{eqnarray}
The non-linear parameters in the basis functions are also optimized. For an unrestricted
optimization of the non-linear parameters in the basis set, the matrix $A_k$ is
expressed as a Cholesky product of a lower triangular matrix and its transposed, $ L_k $ and $ L_k^{\prime} $,
\begin{eqnarray}
A_k = L_k L_k^{\prime}.
\end{eqnarray}
This allows one to optimize the elements of the $L_k$ matrix for each basis function with the freedom
of using all real numbers without overstepping the requirement that each $A_k$ matrix be positive definite
to ensure each ECG is square integrable. In addition to the $L_k$ matrix elements, the
elements of the $\mathbf{s_k}$ vectors are also optimized.

In \cite{HeH+_Wei2012}, the basis set for each PEC point was optimized 
by a simultaneus optimization of all basis functions (global optimizations), 
meaning the energy minimum 
for each point on the curve
depending on all of the non-linear parameters of all basis functions was found 
using an optimization
routine. The TN minimization routine \cite{TN} was ued. 
In this work the basis set for each PEC point is optimized 
by taking one basis function at a time of optimizing its non-linear parameters and
then cycling over all functions several times to get the lowest possible total energy.
In determing the minimum of the energy in terms of the non-linear parameters of a single function 
the same optimization routine as used in the global optimization is used.
The routine employes the energy gradient determing analytically by a direct differenciation 
of the energy expression with respect to the non-linear parameter of the currently
optimized function. This optimization is carried
out with the parameters of the other functions kept frozen. 
Each time the optimized function is changed, the secular equation is solved 
to obtain optimal linear expansion coefficients. After the changes of the total energy
in the optimization of a particula basis function
fall below a certain assumed threshold, the optimization of
the next function starts. This cyclic optimization of the basis set is carried out until
convergence, as judged by comparing the energy changes and 
the norm of the gradient vector with the assumed convergence thresholds,
is reached (see fig. \ref{OptCyc}). 
According to our experience,
carrying out the optimization in this way makes it less
likely that the optimization converges to a local minimum.
Also, this optimization scheme allows to control and prevent linear dependencies
between the basis functions to form, as such dependencies can lower the
numerical accuracy of the calculation and a decrease of the optimization efficiency.

To evaluate the energy, the analytic forms of the Hamiltonian and overlap matrix elements
need to be determined.
This was done before for the types of functions used in the present
work \cite{Theory5, Theory6, Theory7}. The algorithms derived in those
works have been implemented
in the computer code used in the present calculations. 
Additionally, algorithms for
calculating the elements of the analytical gradient vector
being the energy derivatives with respect to the non-linear parameters of
the basis functions were also derived and implemented \cite{Theory5}.
The use of the gradient in the optimization of the non-linear parameters
of the Guassians is a single most important factor allowing for 
achieving high accuracy in calculating the PEC.

The wave functions constructed in this work for HeH$^+$ are very similar to the wave functions
generated in 
\cite{HeH+_Wei2012}. Wave functions from \cite{HeH+_Wei2012} are taken for the PEC points with
internuclear distances between 0.35 and 2.10 a.u. and optimized. The wave functions for the remaining PEC
points up to 100.0 a.u. are constructed by shifting the nuclear coordinates and adjusting
the shifts of the individual basis functions by implementing the Gaussian product theorem as shown
in \cite{GPT}. At this point, the non-linear parameters are reoptimized.

\subsection{Adiabatic and non-Adiabatic corrections}
The adiabatic corrections are calculated in the same way as in \cite{HeH+_Wei2012}.
An exact molecular wave function can be obtained with an expansion 
in a complete basis as:
\begin{eqnarray}
\Psi(\mathbf{r};\mathbf{R}) = \sum_{i}\chi_i(\mathbf{R})\psi_i(\mathbf{r};\mathbf{R}).
\end{eqnarray}

The electronic wave functions, $\psi(\mathbf{r};\mathbf{R})$ depend parametrically on the
nuclear coordinates, $\mathbf{R}$, and form a discreet set of 
wave functions for the PEC in the range
of relevant values of the nuclear configurations. 
When the electronic wave functions, $\psi_i$, are weakly coupled,
variational optimization of $\chi_i$ leads to the following equation being satisfied \cite{Adiabatic1}:
\begin{eqnarray}
\bigg[-\frac{1}{2}\sum_{A=1}^{N_{nuc}}\frac{1}{M_A}\nabla_A^2 + \bigg<\psi_i\bigg|-\frac{1}{2}
\sum_{A=1}^{N_{nuc}}\frac{1}{M_A}
\nabla_A^2\psi_i\bigg>\bigg(\mathbf{R}\bigg) + E_e(\mathbf{R})\bigg]\chi_i = E\chi_i.
\end{eqnarray}
where $E_e(\mathbf{R}$) is the electronic energy 
corresponding to the clamped nucleus Hamiltonian and $N_{nuc}$
is the number of nuclei in the system. The second term,
which depends parametrically on $R$, is called the Born-Oppenheimer diagonal correction \cite{Adiabatic1,
Adiabatic2}. This is the first-order correction to the electronic Born-Oppenheimer energy due to the nuclear motion.
The value of this term at each point, $\mathbf{R}$, is added to the energy 
of the corresponding point of the PEC, $E_e(\mathbf{R})$.
An improved potential energy is obtained. Solving the nuclear Schr$\ddot{o}$dinger equation with
this correction gives improved results for calculated rotational and vibrational transition energies.
From this point forward, the subscript, $i$ will no longer be used to refer to a specific electronic
state because only one is used in this work.

To calculate the adiabatic correction
we implement the procedure of Cenzek and Kutzelnigg \cite{Adiabatic3}. 
The procedure calculates the Born-Oppenheimer adiabatic
diagonal correction. Using integration by parts, it can be shown that
\begin{eqnarray}
\bigg<\psi\bigg|\frac{\partial^2}{\partial Q_{i_A}^2}\bigg|\psi\bigg> = -\bigg<\frac{\partial\psi}{\partial Q_{i_A}}
\bigg|\frac{\partial\psi}{\partial Q_{i_A}}\bigg>,
\end{eqnarray}
where $Q_{i_A}$ is Cartesian coordinate $i$ of nucleus $A$. 
The partial derivative of the electronic wave function
with respect to the nuclear coordinates is calculated numerically as:
\begin{eqnarray}
\frac{\partial\psi}{\partial Q_{i_A}} \approx \frac{\psi(Q_{i_A} + \frac{1}{2}\Delta Q_{i_A}) - 
\psi(Q_{i_A} - \frac{1}{2}\Delta Q_{i_A})}{\Delta Q_{i_A}}.
\end{eqnarray}

Where $\Delta Q_{i_A}$ is some finite displacement of the nuclear coordinate.
The energy for the adiabatic correction is (assuming  that the electronic wave function
is normalized):
\begin{eqnarray}
E_{ad} = \Big<\psi\Big|-\frac{1}{2}\sum_{A=1}^{N_{nuc}}\sum_{i=1}^3\frac{1}{M_A}
\frac{\partial^2}{\partial Q_{i_A}}\Big|\psi\big> = \nonumber\\
\frac{1}{2}\sum_{A=1}^{N_{nuc}}\sum_{i=1}^3\frac{1}{M_A}\Big<\frac{\partial\psi}
{\partial Q_{i_A}}\Big|\frac{\partial\psi}{\partial Q_{i_A}}\Big> = \nonumber\\
\sum_{A=1}^{N_{nuc}}\sum_{i=1}^3\frac{1}{M_A(\Delta Q_{i_A})^2} \bigg[1 -\Big<\psi\Big(
Q_{i_A} - \frac{1}{2}\Delta Q_{i_A}\Big)\Big|\psi\Big(Q_{i_A} + \frac{1}{2}\Delta Q_{i_A}\Big)\Big>
\bigg],
\end{eqnarray}
where $M_A$ is the mass of nucleus $A$. 
Differentiation between isopologues is simply a matter of recalculating
$E_{ad}$ with different nuclear masses.

Shifting the geometry of the system 
slightly and recalculating its wave function, $\psi\Big(Q_{i_A} \pm \frac{1}{2}
\Delta Q_{i_A}\Big)$ requires the utilization of the Gaussian product theorem \cite{ECG_HeH+}. Due to
the fact that numerical differentiation is more effective with a smaller shift of geometrical
coordinates and that
the wave function 
at the sifted geometry is not as well optimized as the wave function 
at the unshifted geometry of the system, $\psi\Big(Q_{i_A}\Big), \Delta Q_{i_A}$
is chosen to be small ($\Delta Q_{i_A} = 10^{-8}$ atomic units). 
Smaller values of $\Delta Q_{i_A}$ can lead
to loss of numerical precision in the calculation.

\subsection{The leading relativistic and quantum electrodynamics corrections}


In high-accuracy quantum-mechanical molecular calculations
the quantum electrodynamics (QED) provides a
general theoretical framework
for calculating relativistic and QED corrections.
At the zeroth-order level in such an approach 
the relativistic effects are represented by
a Hamiltonian derived based on the so--called non--relativistic QED
theory (NRQED) \cite{d17,d18,d19,d20}.
In this theory, the relativistic corrections
appear as quantities proportional to powers of
the fine structure parameter $\alpha$ (where $\alpha = \frac{1}{c}$)
and are determined using the perturbation theory.
In this work the relativistic Hamiltonian is the
electronic relativistic Breit-Pauli Hamiltonian \cite{BS1977}
representing the mass-velocity (MV), one- and two-electron
Darwin (D), orbit-orbit (OO), and spin-spin (SS) relativistic corrections.
The leading relativistic corrections are calculated as the expectation values
of the corresponding relativistic operators using the non-relativistic
BO electronic wave function expanded in terms of
ECGs with shifted centers. The total correction calculated
at each PEC point is a sum of the MV, D, OO, and SS contributions:
\begin{eqnarray}
E_{rel} = E_{rel}^{MV} +  E_{rel}^{D} +  E_{rel}^{OO} +  E_{rel}^{SS}.
\end{eqnarray}
The spin-orbit correction is zero due to the zero total spin
of the electronic state considered in this work. 
The BO electronic non-relativistic wave function
is also used to calculated the leading QED correction
called the Araki-Sucher correction.
The Araki-Sucher QED correction \cite{Be,Ar,Su} provides the leading
QED contribution to the rovibrational transition energies, as can be inferred, for example,
from the calculations concerning the H$_2$ molecule\cite{Jeziorski}.
The other contribution involving the so-called Bethe algorithm,
though significant in its absolute value, changes little with the internuclear
distance and can be neglected when calculating transition energies.

In the present work the calculations of the leading
relativistic and QED corrections are calculated using 
the algorithms recently derived and implemented by our group \cite{ms1,ms2,ms3}
The computer code that includes these algorithms runs in parallel
employing the MPI (message passing interface) protocol.

\section{Results}
\label{Results}
Results

\section{Conclusions}
Conclusions

\section{Acknowledgements}
We are grateful to the University of Arizona Center of Computing and Information Technology
for the use of their computer resources and the El Gato cluster, NSF grant number 1228509.

\newpage

%Key bib
\bibliography{paper}

\newpage

\begin{figure}
\caption{Cartoon representation of sequential, cyclic optimization of individual basis functions.
k = function index. K = Total number of basis functions.}
\includegraphics[width=0.5\linewidth]{OptCycle}
\label{OptCyc}
\end{figure}

\end{document}
