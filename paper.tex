\documentclass[aps,onecolumn]{revtex4}
%\setlength{\oddsidemargin}{-.38in}
%\setlength{\textheight}{9.6in}
\setlength{\textheight}{10.0in}
%\setlength{\textwidth}{6.7in}
%\setlength{\topmargin}{-0.3in}
\usepackage{bm}
\usepackage{array}
\usepackage{color}
\usepackage{rotating}
\usepackage{lscape}
\usepackage{graphicx}
\usepackage{amsmath}
\usepackage{amsfonts}
\usepackage{amssymb}
\usepackage{mathtools}
\usepackage{cases}
\usepackage{tikz,pgfplots}
\pgfplotsset{compat=1.10,width=12cm,
legend style={at={(0.03,0.5)},anchor=west}}
%This defines the interval between lines. To generate
%doublespaced text set it to 2.0
%\renewcommand{\baselinestretch}{1.92}

\renewcommand{\baselinestretch}{1.92}
\newcommand{\tr}{\mathop{\rm tr}\nolimits}
\renewcommand{\d}{\mathop{\rm d \!\,}\nolimits}
\renewcommand{\vec}{\mathop{\rm vec}\nolimits}
\newcommand{\vech}{\mathop{\rm vech}\nolimits}
%Key 1
\begin{document}
\title{Very accurate rovibrational transitions of
HeH$^+$ in the ground electronic state}
\author{
Keith Jones$^{a}$, Monika Stanke,
Ludwik Adamowicz$^{a,b}$}
\affiliation{
$^{a}$Department of Chemistry and Biochemistry,
University of Arizona, Tucson, Arizona 85721, U.S.A.
\nonumber\\
$^{b}$Department of Physics,
University of Arizona, Tucson, Arizona 85721, U.S.A.}

%Key 0 - abstract
\begin{abstract}
HeH$^+$ rovibrational transitions are calculated using a very accurate Born-Oppenheimer
potential energy surface, which is supplemented with adiabatic and non-adiabatic corrections
for nuclear motion, and relativistic corrections for electron motion. Comparisons are made
with leading experimental data.
\end{abstract}

\maketitle

%Key 1 - Intro
\section{Introduction}
\label{Introduction}
HeH$^+$ is the lightest, bound heteronuclear diatomic molecule. It is composed of the
two most abundant elements in the universe. The presence of HeH$^+$ in interstellar space and
planetary nebulae is expected \cite{Astro3, Astro4, Astro5}, and steady state abundances
of the molecule under different conditions have been calculated \cite{Astro1_SS_HeH+} although
it has yet to be observed \cite{Astro2, PES5}.

The existence of HeH$^+$ was confirmed in 1925 by mass spectrometry of discharges
containing He and H \cite{exp3}. Optical measurements would follow decades later.
HeH$^+$ has a permanent dipole moment, and very
accurate transition energies can be measured because of this. In 1979, the first rovibrational
transitions for HeH$^+$  ever measured, 5 transitions in total, were in the fundamental
band and the first hot band of the electronic ground state \cite{exp2} with uncertainties
of 0.002 cm$^{-1}$. In the following decades, many more pure rotational \cite{exp1, exp7, exp9}
and rovibrational \cite{exp4, exp5, exp6, exp8, exp9, exp10} transitions were measured, for
multiple isotopologues \cite{exp1, exp4, exp10}, with many of the later measurements collected
using tunable diode lasers \cite{exp7, exp8, exp10}. There were also transitions between bound
and quasibound levels measured \cite{exp5}.

%1.) 1997 - pure rotational transitions of $^4$HeH$^+$, $^4$HeD$^+$, $^3$HeH$^+$, and
%    $^3$HeD$^+$ in the vibrational ground state \cite{exp1}.
%2.) 1989 - 54 total transitions spanning all four of the common isotopologues in either the
%    fundamental or first hot band. Uncertainties between 0.001 and 0.005 cm$^{-1}$ \cite{exp4}.
%3.) 1983 - Many different rovibrational transitions measured, including bound to quasibound
%    transitions \cite{exp5}.
%4.) 1982 - 9 rovibrational transitions measured in the fundamental band \cite{exp6}.
%5.) 1987 - First time using a tunable diode laser for rotational spectrocopy of molecular ions.
%    J=7<-6 transition in the vibrational ground state of HeH$^+$ \cite{exp7}.
%6.) 1987 - Tunable diode lasers used to measure rovibrational transitions of HeH$^+$ in the
%    fundamental and first hot band \cite{exp8}.
%7.) 1997 - Pure rotational transitions with $\nu$ = 0, 1, 2 and high J (10-21) up to the
%    dissociation limit were measured. 10 new rovibrational levels also measured up to the
%    dissociation limit \cite{exp9}.
%8.) 1992 - Rovibrational transitions in the $\nu$ = 0<-1, $\nu$ = 1<-2, $\nu$ = 2<-3, and
%    $\nu$ = 3<-4 bands were measured for HeH$^+$ and HeD$^+$ with a tunable diode laser
%    \cite{exp10}.

It is important for computational models to keep up with the accuracy of experimental
measurements in order for the unobserved transitions in this system calculated with
those models to be assumed correct. The most common way to model diatomics is to construct
a potential energy surface, for the internuclear coordinate, and then solving the radial
Schr$\ddot{o}$dinger equation to acquire the rovibrational levels.

The first potential energy surface for HeH$^+$ that was done accurately used a 64-term
generalized James-Coolidge expansion for the wavefunction \cite{PES7}, which allowed the
approximation of the equilibrium bond length and the dissociation energy. Later, a potential
energy surface of HeH$^+$ between 0.9 and 9 au was constructed using an 83-term variational
wave function expansion in elliptic coordinates with linear and non-linear variational
parameters \cite{PES3, PES4}, which was later improved by using a larger basis set and
adiabatic corrections \cite{PES8}. A potential energy curve for HeH$^+$ was fitted using
experimental transitions and 19 tunable parameters \cite{PES2_exptrans_fit}. Recently,
a very accurate Born-Oppenheimer (BO) potential energy surface was constructed for HeH$^+$
in a basis of "asymptotically correct generalized Heitler-London functions"\cite{PES6},
followed by adiabatic, non-adiabatic, relativistic, and radiative corrections, which allowed
the accurate calculation of the rovibrational transitions \cite{PES5}.

Explicitly correlated Gaussian functions (ECGs) have been used to describe HeH$^+$ with accurate
single point calculations carried out for different electronic states \cite{ECG_HeH+} at
the equilibrium distance and a potential energy surface developed in 2012 which incorporated
adiabatic and non-adiabatic corrections \cite{HeH+_Wei2012}.

ECGs have also been used to describe HeH$^+$ without assuming the Born-Oppenheimer approximation.
All rotationless vibrational states in the ground electronic have been calculated \cite{Theory3}
with relativistic and radiative corrections \cite{Theory4, Theory2}. This has also been done
with complex basis functions \cite{Theory1}. These wavefunctions are probabilistic functions
of both electrons and nuclei and can be used to glean insight into molecular structure that is
not possible from the BO description.

%MAYBE WHEN DISCUSSING WORK THAT HAS BEEN DONE, MENTION THE ORDER TO WHICH THE REL. RAD. CORRECTIONS
%HAVE BEEN DONE FOR EACH PARTICULAR STUDY.
%MAYBE MENTION ACCURACY OF DIFFERENT CALCULATSION IN CM-1

\section{Methodology}
\label{MTD}
Methodology

\section{Results}
\label{Results}
Results

\section{Conclusions}
Conclusions

\section{Acknowledgements}
We are grateful to the University of Arizona Center of Computing and Information Technology
for the use of their computer resources and the El Gato cluster, NSF grant number 1228509.

\newpage

%Key bib
\bibliography{paper}

\newpage

\end{document}
